
% time: Thursday 09:00
% URL: https://pretalx.com/fossgis2019/talk/VM37SR/

\noindent%
\newTimeslot{09:00}
\abstractAudimax{%
  Claas Leiner, Bernhard Ströbl%
}{%
  QGIS  das GIS mit unbegrenzten Darstellungsmöglichkeiten%
}{%
}{%
  Bei der Kartendarstellung im QGIS wird es immer schwieriger,

  auf wirklich unüberwindliche Grenzen zu stoßen


  Mit Hilfe des verschachtelbaren Symol-Layer-Konzeptes, einiger spezieller Symbol-Layer-Typen und
  des Geometriegenerators, lassen sich auch ungewöhnliche  Flächensignaturen und exzentrische Ideen
  umsetzen .  Regelbasierte  Darstellung und datendefinierte  Übersteuerungen machen die Inhalte
  sämtlicher Attribute gleichzeitig für die Darstellung nutzbar. Ausnahmen werden einfach definierba%
}


%%%%%%%%%%%%%%%%%%%%%%%%%%%%%%%%%%%%%%%%%%%

% time: Thursday 09:00
% URL: https://pretalx.com/fossgis2019/talk/ZZ7EJH/

\noindent%

\abstractMathe{%
  Richard Figura, Dr. Alexander Willner, Michael Martin%
}{%
  Smarte Daten im Knowledge Graph, die Grundlage einer zukunftssicheren Bereitstellung Offener Daten.%
}{%
}{%
  Offene Daten sind einer der wichtigsten Rohstoffe der digitalen Welt, mit wachsender
  wirtschaftlicher und gesellschaftlicher Bedeutung. Trotz zahlreicher Bemühungen konnten
  prognostizierte Mehrwerte noch nicht erreicht werden, was unter anderem auf eine unvollständige
  Vernetzung der Daten zurückzuführen ist. In diesem Vortrag werden Technologien und Prozesse
  vorgestellt, um Daten zu einem öffentlichen verfügbaren Knowledge Graph hinzuzufügen und dort mit
  Daten anderer Quellen zu verknüpfen.%
}


%%%%%%%%%%%%%%%%%%%%%%%%%%%%%%%%%%%%%%%%%%%

% time: Thursday 09:00
% URL: https://pretalx.com/fossgis2019/talk/USA8LS/

\noindent%

\abstractPhysik{%
  Doris Schuller, David Kirschheuter%
}{%
  Wie Archäolog*innen GIS (nicht) nutzen%
}{%
}{%
  In der archäologischen Forschung werden GIS bereits erfolgreich eingesetzt. Wie sieht es im
  Bereich Grabungsdokumentation aus?%
}


%%%%%%%%%%%%%%%%%%%%%%%%%%%%%%%%%%%%%%%%%%%

% time: Thursday 09:30
% URL: https://pretalx.com/fossgis2019/talk/79VWTC/

\noindent%
\newTimeslot{09:30}
\abstractMathe{%
  Andreas Krumtung%
}{%
  Quo Vadis Open Data – Geoportale von Bund und Ländern auf dem Prüfstein%
}{%
}{%
  Der Beitrag gibt einen Überblick über die Geoportallandschaft des Bundes und der Länder in
  Deutschland und zeigt auf, welche Hausaufgaben Bund und Länder noch haben, wenn sie
  funktionierende Datenökosysteme um ihre Portale herum etablieren wollen.%
}


%%%%%%%%%%%%%%%%%%%%%%%%%%%%%%%%%%%%%%%%%%%

% time: Thursday 09:30
% URL: https://pretalx.com/fossgis2019/talk/S7NKWQ/

\noindent%

\abstractPhysik{%
  Christian Trapp%
}{%
  Tachy2GIS: Mit der Totalstation zeichnen%
}{%
}{%
  Tachy2GIS ist ein QGIS-Plugin zur direkten Erfassung dreidimensionaler Geometrien mit dem
  Tachymeter%
}


%%%%%%%%%%%%%%%%%%%%%%%%%%%%%%%%%%%%%%%%%%%

% time: Thursday 10:00
% URL: https://pretalx.com/fossgis2019/talk/PLRQCU/

\noindent%
\newTimeslot{10:00}
\abstractAudimax{%
  Johannes Kröger%
}{%
  Kartografie-Rezepte für die Experimentalküche%
}{%
}{%
  Letztes Jahr gab es "`5-Minuten-Kartografie-Rezepte aus der QGIS-Trickkiste"' als Lightning-Talk,
  ein wilder Ritt durch einige Spielereien ohne Zeit für Erklärungen. In dieser Demo-Session werden
  wieder ähnlich interessante, ausgefallene, praktische oder künstlerische kartografische Kniffe
  gezeigt und die Herangehensweise diesmal *ausführlich* erläutert.%
}


%%%%%%%%%%%%%%%%%%%%%%%%%%%%%%%%%%%%%%%%%%%

% time: Thursday 10:00
% URL: https://pretalx.com/fossgis2019/talk/9FTWCX/

\noindent%

\abstractMathe{%
  Armin Retterath%
}{%
  Der neue Standard für Darstellungsdienste in Deutschland%
}{%
}{%
  Im Rahmen der Veranstaltung wird der neue Standard für die interoperable Bereitstellung von WMS
  und WMTS Diensten innerhalb der Geodateninfrastruktur Deutschland (GDI-DE) vorgestellt, und es
  wird anhand praktischer Beispiele erläutert, welche Konsequenzen sich für die bereitstellenden
  Institutionen ergeben.%
}


%%%%%%%%%%%%%%%%%%%%%%%%%%%%%%%%%%%%%%%%%%%

% time: Thursday 10:00
% URL: https://pretalx.com/fossgis2019/talk/QPNPK7/

\noindent%

\abstractPhysik{%
  Prof. Dr. Marco Block-Berlitz, Hendrik Rohland, Dr. Christina Franken%
}{%
  QGIS als Forschungswerkzeug in der Archäologie - Anwendungen bei der Mongolisch-Deutschen Orchon-Expedition%
}{%
}{%
  Im Rahmen der Mongolisch-Deutschen Orchon-Expedition wird QGIS als Standardwerkzeug für die
  Planung und Durchführung von Kampagnen eingesetzt. Im Besonderen soll die Befliegungskampagne im
  September 2018 vorgestellt werden, bei der innerhalb von nur 5 Tagen mehr als 50 Quadratkilometer
  aufgenommen und rekonstruiert wurden. Die Herausforderung der Logistik einer solchen Kampagne
  erfordert eine akurate Planung im Vorfeld und vor Ort. Die wichtige Rolle von QGIS wird gezeigt.%
}


%%%%%%%%%%%%%%%%%%%%%%%%%%%%%%%%%%%%%%%%%%%

% time: Thursday 11:00
% URL: https://pretalx.com/fossgis2019/talk/877HVT/

\noindent%
\newTimeslot{11:00}
\abstractAudimax{%
  Jörg Thomsen%
}{%
  Vergleich QGIS-Server, Geoserver und MapServer%
}{%
}{%
  Wo liegen die Unterschiede zwischen den drei Engines, was sind Gemeinsamkeiten, Schnittstellen und
  womöglich spezifische Einsatzgebiete?%
}


%%%%%%%%%%%%%%%%%%%%%%%%%%%%%%%%%%%%%%%%%%%

% time: Thursday 11:00
% URL: https://pretalx.com/fossgis2019/talk/GY7EGG/

\noindent%

\abstractMathe{%
  Axel Lorenzen-Zabel%
}{%
  OpenGeoEdu - mit offenen Daten lernen%
}{%
}{%
  Wir stellen Ihnen den ersten offenen Online-Kurs mit dem Kernthema OpenData vor und geben
  Einblicke in die Ergebnisse unseres ersten Semesters%
}


%%%%%%%%%%%%%%%%%%%%%%%%%%%%%%%%%%%%%%%%%%%

% time: Thursday 11:00
% URL: https://pretalx.com/fossgis2019/talk/CJGV7F/

\noindent%

\abstractPhysik{%
  Daniel Kastl%
}{%
  Dotloom - grosse Point-Cloud-Daten im Distributed Web%
}{%
}{%
  Dotloom ist ein Open-Source-Projekt und ermöglicht die Synchronisation, Replikation, Indexierung
  und Verarbeitung von Terabyte an Geodaten mit Peer-to-Peer-Technologien. Aufbauend auf dem "`DAT
  Projekt"' erweitert Dotloom die Funktionalität speziell zur Verwaltung, Abfrage und Visualisierung
  von Point-Cloud-Daten.%
}


%%%%%%%%%%%%%%%%%%%%%%%%%%%%%%%%%%%%%%%%%%%

% time: Thursday 11:30
% URL: https://pretalx.com/fossgis2019/talk/8VEUHT/

\noindent%
\newTimeslot{11:30}
\abstractAudimax{%
  Jan Suleiman, Christian Mayer, Kai, Daniel Koch%
}{%
  GeoStyler: ein generischer grafischer Stileditor für Geodaten%
}{%
}{%
  GeoStyler ist eine react-basierte OpenSource UI-Bibliothek zur Erstellung von Stileditoren für
  WebGIS-Anwendungen. Der Vortrag stellt den aktuellen Stand und mögliche nächste Schritte vor.%
}


%%%%%%%%%%%%%%%%%%%%%%%%%%%%%%%%%%%%%%%%%%%

% time: Thursday 11:30
% URL: https://pretalx.com/fossgis2019/talk/NBXCC7/

\noindent%

\abstractMathe{%
  Dr. Sebastian Meier%
}{%
  Mehrwert für Bürger*innen schaffen%
}{%
}{%
  Wie können offene räumliche Daten zum Nutzen der Zivilgesellschaft sinnvoll aufbereitet werden?
  Das Ideation \& Prototyping Lab der gemeinnützigen Technologiestiftung Berlin gibt Einblicke in
  räumliche Open Data Anwendungen für die Berliner Bürger*innen.%
}


%%%%%%%%%%%%%%%%%%%%%%%%%%%%%%%%%%%%%%%%%%%

% time: Thursday 11:30
% URL: https://pretalx.com/fossgis2019/talk/E8QA8U/

\noindent%

\abstractPhysik{%
  Jelto Buurman%
}{%
  Verarbeitung von DGM Daten und Laserscandaten mit QGIS%
}{%
}{%
  Ausgehend von einem kurzen Abriss über Laserscandaten, wird dargestellt, wie diese Daten in QGIS
  aufbereitet und verarbeitet werden können. Es werden einige Beispiele für die Anwendung von
  Laserscandaten vorgestellt.

  Abschließend wir anhand einer Live Präsentation die hervorragende Performance gezeigt.%
}


%%%%%%%%%%%%%%%%%%%%%%%%%%%%%%%%%%%%%%%%%%%

% time: Thursday 12:00
% URL: https://pretalx.com/fossgis2019/talk/8MZZGX/

\noindent%
\newTimeslot{12:00}
\abstractAudimax{%
  Oliver Fink, Johannes Lauer%
}{%
  HERE XYZ \& QGIS - Ein neuer OpenSource Map Hub made by HERE%
}{%
}{%
  HERE XYZ ist eine Sammlung von Open-Source-Tools um die arbeiten mit Geodaten zu vereinfachen. Die
  Basis ist der XYZ Hub: Es ist ein Echtzeit, Cloud-basierter Location Hub zum Auffinden, Speichern,
  Laden, Bearbeiten und Veröffentlichen von privaten oder öffentlichen geographischen Daten. Wir
  zeigen die Vorteile dieser Tools, insbesondere die Interoperabilität mit etablierten Open-Source-
  Lösung wie QGIS.%
}


%%%%%%%%%%%%%%%%%%%%%%%%%%%%%%%%%%%%%%%%%%%

% time: Thursday 12:00
% URL: https://pretalx.com/fossgis2019/talk/AP8QL3/

\noindent%

\abstractMathe{%
  Uwe Raudszus, Dr. Roland Olbricht, Martin Kucharzewski%
}{%
  Barrierefreies Fußgängerrouting für Dortmund%
}{%
}{%
  Die Stadt Dortmund entwickelt für Fußgänger ein barrierefreies Routingsystem von Tür zu Tür für
  Menschen, die blind, sehbehindert, hörgeschädigt oder mobilitätseingeschränkt sind


  Die Lösung bleibt nachhaltig durch Integration in die App des lokalen Verkehrsverbundes und für
  Dritte verfügbar durch Aufbau auf OpenStreetMap. So können durch die Koordinierung von mehreren
  Akteuren bestehende Strukturen mit geringem Aufwand in jeder Hinsicht effizienter genutzt werden.%
}


%%%%%%%%%%%%%%%%%%%%%%%%%%%%%%%%%%%%%%%%%%%

% time: Thursday 12:00
% URL: https://pretalx.com/fossgis2019/talk/BL9N8S/

\noindent%

\abstractPhysik{%
  Mira Kattwinkel%
}{%
  GRASS GIS und R zur Datenaufbereitung für räumliche Regressionsmodelle%
}{%
}{%
  Das R-Paket 'openSTARS' ermöglicht die Aufbereitung von Geodaten zur Erstellung räumlicher
  Regressionsmodelle und bietet so eine freie, auf R und GRASS GIS basierende Alternative zur ArcGIS
  Toolbox 'STARS'.%
}


%%%%%%%%%%%%%%%%%%%%%%%%%%%%%%%%%%%%%%%%%%%

% time: Thursday 13:30
% URL: https://pretalx.com/fossgis2019/talk/9PEAWQ/

\noindent%
\newTimeslot{13:30}
\abstractAudimax{%
  Andreas Neumann%
}{%
  QGIS 3D%
}{%
}{%
  Seit QGIS 3.0 gibt es erste echte 3D-Visualisierungsmöglichkeiten im QGIS-Kern ohne Plugins oder
  Drittsoftware installieren zu müssen. Die Präsentation zeigt was im Bereich 3D Visualisierung
  bereits möglich ist und wo es noch Probleme gibt.%
}


%%%%%%%%%%%%%%%%%%%%%%%%%%%%%%%%%%%%%%%%%%%

% time: Thursday 13:30
% URL: https://pretalx.com/fossgis2019/talk/AYKS9Z/

\noindent%

\abstractMathe{%
  Torsten Brassat%
}{%
  SHOGun-QGIS-Integration: WebGIS-Applikationen vom Desktop administrieren%
}{%
}{%
  Das OpenSource QGIS Plugin "`SHOGun-Editor"' zeigt, wie SHOGun-basierte WebGIS-Applikationen in QGIS
  im Hinblick auf Hinzufügen und Stylen von Layern und Applikationen administriert werden können.%
}


%%%%%%%%%%%%%%%%%%%%%%%%%%%%%%%%%%%%%%%%%%%

% time: Thursday 13:30
% URL: https://pretalx.com/fossgis2019/talk/WSALG8/

\noindent%

\abstractPhysik{%
  Nikolai Janakiev%
}{%
  Data Science mit OpenStreetMap%
}{%
}{%
  Data Science ist ein populäres Schlagwort, das schon vielerlei Bereiche befallen hat, nicht
  zuletzt die Welt der Geoinformatik. Hier geht es darum, wie man gängige Methoden von Data Science
  auf OpenStreetMap Daten mithilfe von Open Source Werkzeugen anwenden kann und daraus neue
  Einblicke erzeugen kann.%
}


%%%%%%%%%%%%%%%%%%%%%%%%%%%%%%%%%%%%%%%%%%%

% time: Thursday 14:00
% URL: https://pretalx.com/fossgis2019/talk/HBKJM9/

\noindent%
\newTimeslot{14:00}
\abstractAudimax{%
  Marco Bernasocchi%
}{%
  QField - der mobile QGIS Alleskönner%
}{%
}{%
  QGIS is efficient and comfortable in everyday office life. However, data collection often begins
  on the field. Whether in shiver or sunshine, working outdoors requires a solution that is
  optimized for mobile devices. QField is the perfect companion of QGIS.

  In this extended demo we will show you the most important features currently available%
}


%%%%%%%%%%%%%%%%%%%%%%%%%%%%%%%%%%%%%%%%%%%

% time: Thursday 14:00
% URL: https://pretalx.com/fossgis2019/talk/ACCU9K/

\noindent%

\abstractMathe{%
  Mathias Gröbe%
}{%
  Aktuelle Möglichkeiten der kartographischen Reliefdarstellung%
}{%
}{%
  Ideen für eine bessere Geländevisualisierung mit aktuellen Methoden und altbekanntem Wissen: Ein
  Überblick über Potential, Fehlerquellen und Möglichkeiten.%
}


%%%%%%%%%%%%%%%%%%%%%%%%%%%%%%%%%%%%%%%%%%%

% time: Thursday 14:00
% URL: https://pretalx.com/fossgis2019/talk/ZTRDY9/

\noindent%

\abstractPhysik{%
  Dr. Roland Olbricht%
}{%
  Wie aktuell sind OpenStreetMap-Daten?%
}{%
}{%
  Wenn die Qualität von OpenStreetMap-Daten diskutiert wird, ist eine große offene Frage, ob wir
  OpenStreetMap-Mapper die Daten dauerhaft aktuell halten können


  In dem Vortrag wird es einerseits um Wekzeuge gehen, wie man abschätzen kann, welche Daten wohl
  wie aktuell sind, um Zweifel bei Mappern und Außenstehenden ausräumen zu können. Andererseits wird
  es um Methodiken gehen, wie man das Aktuell-Halten bequem und attraktiv machen kann.%
}


%%%%%%%%%%%%%%%%%%%%%%%%%%%%%%%%%%%%%%%%%%%

% time: Thursday 14:30
% URL: https://pretalx.com/fossgis2019/talk/X8VMWG/

\noindent%
\newTimeslot{14:30}
\abstractAudimax{%
  Numa Gremling%
}{%
  Leaflet: komfortabel Webmaps erstellen%
}{%
}{%
  Leaflet ist eine der momentan meistbenutzten und beliebtesten Bibliotheken um Webmaps zu
  erstellen. In diesem Vortrag lernen Sie wieso.%
}


%%%%%%%%%%%%%%%%%%%%%%%%%%%%%%%%%%%%%%%%%%%

% time: Thursday 14:30
% URL: https://pretalx.com/fossgis2019/talk/7KP7RU/

\noindent%

\abstractMathe{%
  Arne Schubert%
}{%
  Geschichten aus der Docker-Trickkiste%
}{%
}{%
  In der Zeit des Lighning Talks sollen so viele [Docker](https://docker.com)-Befehle wie möglic

  vorgestellt werden, die das Arbeiten mit Docker erleichtern.%
}


%%%%%%%%%%%%%%%%%%%%%%%%%%%%%%%%%%%%%%%%%%%

% time: Thursday 14:30
% URL: https://pretalx.com/fossgis2019/talk/TLJ8AX/

\noindent%

\abstractPhysik{%
  Jochen Topf%
}{%
  Osmoscope - Ein neues QA-Tool für OpenStreetMap%
}{%
}{%
  Osmoscope ist ein neues Tool zur Qualitätssicherung von OSM-Daten. Aufbereitung der Daten und
  Webclient sind komplett getrennt. Jeder kann einfach eigene Layer im Web veröffentlichen und
  Mapper können sie per Mausklick in Osmoscope einbinden.%
}


%%%%%%%%%%%%%%%%%%%%%%%%%%%%%%%%%%%%%%%%%%%

% time: Thursday 14:35
% URL: https://pretalx.com/fossgis2019/talk/GMFJL3/

\noindent%
\newTimeslot{14:35}
\abstractMathe{%
  David Arndt%
}{%
  OGC Diensteverwaltung über SVN%
}{%
}{%
  Zur Verwaltung der UMN Mapserver und QGIS-Dienste wird beim Regionalverband Ruhr Subversion
  genutzt. Über post-commit Hooks werden für jeden Dienst eine Apache-Konfiguration erzeugt und auf
  die Reverse-Proxy verteilt. Zusätzlich können Berechtigungen für den Zugriff auf die Dienste mit
  angegeben werden.%
}


%%%%%%%%%%%%%%%%%%%%%%%%%%%%%%%%%%%%%%%%%%%

% time: Thursday 14:40
% URL: https://pretalx.com/fossgis2019/talk/JRYKSS/

\noindent%
\newTimeslot{14:40}
\abstractMathe{%
  Alexander Matheisen%
}{%
  OpenLayers Editor 2%
}{%
}{%
  [OLE2](https://github.com/geops/ole2) ist die Neuentwicklung des bewährten OpenLayers Editors
  (OLE). Die Open-Source-Bibliothek stellt einfach zu verwendende Werkzeuge für die Erfassung und
  Bearbeitung von Geodaten bereit.%
}


%%%%%%%%%%%%%%%%%%%%%%%%%%%%%%%%%%%%%%%%%%%

% time: Thursday 14:45
% URL: https://pretalx.com/fossgis2019/talk/N9M79P/

\noindent%
\newTimeslot{14:45}
\abstractMathe{%
  Astrid Emde, Jörg Thomsen%
}{%
  Auf dem Weg nach QualityLand oder schon mittendrin?%
}{%
}{%
  Die Frage beantwortet jede/r anders. Hintergründe und Fakten zu QualityLand. Mit Bezug auf das
  Buch von Marc Uwe Kling und Fokus auf der Bedeutung von Geodaten in Zeiten von QualityLand und
  Parallelen zum Alltag und Nirwana.%
}


%%%%%%%%%%%%%%%%%%%%%%%%%%%%%%%%%%%%%%%%%%%

% time: Thursday 14:50
% URL: https://pretalx.com/fossgis2019/talk/GDCVPP/

\noindent%
\newTimeslot{14:50}
\abstractMathe{%
  Christoph Jung%
}{%
  Offline-MapMatching - QGIS-Plugin zum Abgleich einer Trajektorie mit einem Wegenetz%
}{%
}{%
  Das Plugin Offline-MapMatching stellt die erste Erweiterung für QGIS dar, mit der eine Trajektorie
  mit einem Wegenetz auf Basis eines Hidden Markov Models und des Viterbi-Algorithmus abgeglichen
  werden kann.%
}


%%%%%%%%%%%%%%%%%%%%%%%%%%%%%%%%%%%%%%%%%%%

% time: Thursday 15:00
% URL: https://pretalx.com/fossgis2019/talk/RB3CUX/

\noindent%
\newTimeslot{15:00}
\abstractOther{%
  Dominik Helle%
}{%
  Fototermin vom Haupteingang des Z-Gebäudes%
}{%
}{%
  Fototermin vom Haupteingang des Z-Gebäudes Fototermin vom Haupteingang des Z-Gebäudes Fototermin
  vom Haupteingang des Z-Gebäudes Fototermin vom Haupteingang des Z-Gebäudes%
}


%%%%%%%%%%%%%%%%%%%%%%%%%%%%%%%%%%%%%%%%%%%

% time: Thursday 15:30
% URL: https://pretalx.com/fossgis2019/talk/KRBJGJ/

\noindent%
\newTimeslot{15:30}
\abstractAudimax{%
  Marc Jansen, Andreas Hocevar%
}{%
  OpenLayers: Stand und aktuelle Entwicklungen%
}{%
}{%
  Im Vortrag wird der aktuelle Stand und potentielle künftige Weiterentwicklung der weitverbreiteten
  JavaScript Bibliothek OpenLayers vorgestellt.%
}


%%%%%%%%%%%%%%%%%%%%%%%%%%%%%%%%%%%%%%%%%%%

% time: Thursday 15:30
% URL: https://pretalx.com/fossgis2019/talk/LDUZG3/

\noindent%

\abstractMathe{%
  Michael Reichert%
}{%
  Wenn Firmen mappen%
}{%
}{%
  In diesem Vortrag berichtet der Autor seine Erfahrungen mit Kunden, die Schulungen zum Beitragen
  und Verbessern von OpenStreetMap-Daten in Anspruch genommen haben


  Welche kulturelle Unterschiede und Anpassungsschwierigkeiten zwischen der geschäftlichen Welt und
  einem offenen Projekt bestehen? Was sollte bei der Regulierung kommerziell motivierter
  Datenerfassung berücksichtigt werden? Welche Potentiale bietet eine Integration "`kommerzieller"
  Beitragender für OSM in Deutschland?%
}


%%%%%%%%%%%%%%%%%%%%%%%%%%%%%%%%%%%%%%%%%%%

% time: Thursday 15:30
% URL: https://pretalx.com/fossgis2019/talk/VZF97U/

\noindent%

\abstractPhysik{%
  Peter Neubauer%
}{%
  FOSS und GIS Integrationen mit Mapillary%
}{%
}{%
  In diesem Vortrag wird auf die Daten-pipeline des Mapillary - Projektes eingegangen. Es werden
  auch die unterschiedlichen FOSS Komponenten beleuchtet (OpenSfM,MapillaryJS,  iOS SDK, Python
  tools, JOSM/iD - integrationen)%
}


%%%%%%%%%%%%%%%%%%%%%%%%%%%%%%%%%%%%%%%%%%%

% time: Thursday 16:00
% URL: https://pretalx.com/fossgis2019/talk/EM3QFE/

\noindent%
\newTimeslot{16:00}
\abstractAudimax{%
  Daniel Koch, Marc Jansen%
}{%
  3D-Geoapplikationen im Browser: Überblick und Erfahrungen%
}{%
}{%
  Webbasierte OpenSource 3D-Applikationen mit geografischem Bezug sind bereits seit vielen Jahren
  technisch möglich, wie einige projektlösunge zeigen. Der Vortrag wird einige solcher Lösungen
  vorstellen und auch auf künftige Entwicklungen / neue Ansätze / Bibliotheken eingehen.%
}


%%%%%%%%%%%%%%%%%%%%%%%%%%%%%%%%%%%%%%%%%%%

% time: Thursday 16:00
% URL: https://pretalx.com/fossgis2019/talk/TCLACK/

\noindent%

\abstractMathe{%
  Pascal Neis%
}{%
  Untersuchung zum Bezahlten- und Organisierten-Mapping im OpenStreetMap Projekt - Zahlen und Fakten?%
}{%
}{%
  In diesem Vortrag wird ein prototypischer Ansatz präsentiert, wie möglicherweise bezahlte oder
  organisierte Mapper im OSM Projekt erkannt werden können. Der zweite Teil des Vortrags widmet sich
  der Untersuchung der Beitragenden.%
}


%%%%%%%%%%%%%%%%%%%%%%%%%%%%%%%%%%%%%%%%%%%

% time: Thursday 16:00
% URL: https://pretalx.com/fossgis2019/talk/H7399S/

\noindent%

\abstractPhysik{%
  Robert Klemm%
}{%
  Drohne im WebGIS - wie kommen Drohen-Bilddaten ins das WebGIS, am Beispiel von OpenDroneMap%
}{%
}{%
  Im Vortrag möchte ich zeigen, wie man in wenigen Schritten aus den Drohen-Rohbilddaten ein
  nutzbares WebGIS-Projekt erstellen kann. Dabei wird zusammenfassend auf das OpenDroneMap-Projekt
  eingegangen und welche Tools wichtig sind um ein WebGIS zu installieren und zu benutzen.%
}


%%%%%%%%%%%%%%%%%%%%%%%%%%%%%%%%%%%%%%%%%%%

% time: Thursday 16:30
% URL: https://pretalx.com/fossgis2019/talk/EZGELZ/

\noindent%
\newTimeslot{16:30}
\abstractAudimax{%
  Pirmin Kalberer%
}{%
  QWC2 Viewer für QGIS Server mit Micro-Service Architektur%
}{%
}{%
  Der QGIS Webclient 2 (QWC2) ist ein moderner Kartenclient, der auf die Publikation von Karten mit
  QGIS Server spezialisiert ist. Dank dem Einsatz von Micro-Services ist er sowohl für die
  Erstellung einfacher In-House Clients, als auch für umfangreiche Lösungen in Enterprise-
  Infrastrukturen geeignet.%
}


%%%%%%%%%%%%%%%%%%%%%%%%%%%%%%%%%%%%%%%%%%%

% time: Thursday 16:30
% URL: https://pretalx.com/fossgis2019/talk/T83MYX/

\noindent%

\abstractMathe{%
  Frederik Ramm%
}{%
  OpenStreetMap-Vandalismus für Datennutzer – Arten, Häufigkeit, Schutzstrategien%
}{%
}{%
  Viele OpenStreetMap-Nutzer sind erstaunt, wenn sie hören, dass jede(r) einfach alles ändern kann.
  Geht da nicht ständig etwas kaputt? Treiben da nicht Teenager ihren Schabernack mit den heiligen
  Geodaten? Dieser Vortrag analysiert die Risiken und gibt Handlungsempfehlungen.%
}


%%%%%%%%%%%%%%%%%%%%%%%%%%%%%%%%%%%%%%%%%%%

% time: Thursday 16:30
% URL: https://pretalx.com/fossgis2019/talk/AAS7PH/

\noindent%

\abstractPhysik{%
  Till Adams%
}{%
  Vom Luftbild zur Trassenplanung%
}{%
}{%
  Im Vortrag zeigen wir die Verarbeitung der OpenData des Bundeslands NRW für die Planung von
  potentiellen Trassen über Kostenoberflächen mittels KI-basierte Luftbildauswertung bis hin zum
  WebGIS für Planer.%
}


%%%%%%%%%%%%%%%%%%%%%%%%%%%%%%%%%%%%%%%%%%%

% time: Thursday 17:00
% URL: https://pretalx.com/fossgis2019/talk/ZVFKPQ/

\noindent%
\newTimeslot{17:00}
\abstractAudimax{%
  Astrid Emde%
}{%
  Mapbender - Neues aus dem Projekt%
}{%
}{%
  Mapbender ist eine Software zur einfachen Erstellung von WebGIS Anwendungen. Über ein paar Klicks
  können mit dem webbasierten Administrations-Backend individuelle Anwendungen erstellt werden, eine
  Benutzer- und Gruppenverwaltung mit der Möglichkeit Rechte zuzuweisen.


  Der Vortrag geht vor allem auf die neuen Komponenten in Mapbender ein und stellt diese vor.


  Außerdem soll auf die Neuerungen in der Software eingegangen werden. Im letzten Jahr ist viel
  passiert, so dass es Einiges zu präsentie%
}


%%%%%%%%%%%%%%%%%%%%%%%%%%%%%%%%%%%%%%%%%%%

% time: Thursday 17:00
% URL: https://pretalx.com/fossgis2019/talk/SNLZWZ/

\noindent%

\abstractMathe{%
  Christopher Lorenz%
}{%
  osm\_address\_db - Ein Statusbericht zu den Adressdaten%
}{%
}{%
  Im Projekt osm\_address\_db hat sich einiges bewegt. Es wurde schon auf den vergangenen Konferenzen
  vorgestellt.%
}


%%%%%%%%%%%%%%%%%%%%%%%%%%%%%%%%%%%%%%%%%%%

% time: Thursday 17:00
% URL: https://pretalx.com/fossgis2019/talk/TLKKGE/

\noindent%

\abstractPhysik{%
  Christian Strobl%
}{%
  Download / Schnittstellen für Copernicus-Daten mit CODE-DE%
}{%
}{%
  von Astrid angelegt für Christian: Siehe Mail am PKO 23.11.2018. Nicht als Demosession sondern 30
  min Vortrag eingefügt.%
}


%%%%%%%%%%%%%%%%%%%%%%%%%%%%%%%%%%%%%%%%%%%

% time: Thursday 17:05
% URL: https://pretalx.com/fossgis2019/talk/9TTXUN/

\noindent%
\newTimeslot{17:05}
\abstractMathe{%
  Wiard Frühling%
}{%
  Kosten-Distanz-Analyse zur Berechnung von Snapping Punkten auf dem Straßennetz%
}{%
}{%
  Es wurde eine Kosten-Distanz-Analyse zur Berechnung von Road Snapping Punkten für Gebäude im
  ländlichen Raum durchgeführt. Die Ergebnisse  dieser Analyse wurden mit Ergebnissen klassischer
  Methoden des Road Snappings verglichen.%
}


%%%%%%%%%%%%%%%%%%%%%%%%%%%%%%%%%%%%%%%%%%%

% time: Thursday 17:10
% URL: https://pretalx.com/fossgis2019/talk/DY87SF/

\noindent%
\newTimeslot{17:10}
\abstractMathe{%
  Robert Klemm%
}{%
  Neues zur LKW-Maut in Deutschland; Probleme in OSM?%
}{%
}{%
  Kurze Übersicht welche Erneuerungen im LKW-Maut-Gesetz dazugekommen sind und welche Folgen es für
  das OSM-Maut-Tagging-Schemata hat.%
}


%%%%%%%%%%%%%%%%%%%%%%%%%%%%%%%%%%%%%%%%%%%

% time: Thursday 17:15
% URL: https://pretalx.com/fossgis2019/talk/SAAQTC/

\noindent%
\newTimeslot{17:15}
\abstractMathe{%
  Tobias Knerr%
}{%
  OSM in 3D: Facelift für OSM2World%
}{%
}{%
  OSM2World ermöglicht die Erzeugung von 3D-Modellen aus OpenStreetMap-Daten. Dank moderner
  Webtechnologien ist es jetzt auch komfortabel im Browser nutzbar.%
}


%%%%%%%%%%%%%%%%%%%%%%%%%%%%%%%%%%%%%%%%%%%

% time: Thursday 17:30
% URL: https://pretalx.com/fossgis2019/talk/YH8XUW/

\noindent%
\newTimeslot{17:30}
\abstractAudimax{%
  Jörg Thomsen%
}{%
  Mapbender Anwendertreffen%
}{%
}{%
  Zu diesem Treffen sind Anwender und Entwickler der WebGIS Client Suite Mapbender eingeladen. Aber
  es auch alle anderen Interessierten willkommen!%
}


%%%%%%%%%%%%%%%%%%%%%%%%%%%%%%%%%%%%%%%%%%%

% time: Thursday 17:30
% URL: https://pretalx.com/fossgis2019/talk/7Z3TNX/

\noindent%

\abstractRecht{%
  Torsten Friebe, Dirk Stenger%
}{%
  deegree für INSPIRE Anwendertreffen%
}{%
}{%
  Zum Anwendertreffen sind Anwender und Entwickler herzlich eingeladen, die INSPIRE Netzwerkdienste
  mit dem [OSGeo Projekt deegree](https://www.deegree.org/) bereits umsetzen oder dieses für die
  Zukunft planen.%
}


%%%%%%%%%%%%%%%%%%%%%%%%%%%%%%%%%%%%%%%%%%%

% time: Thursday 18:00
% URL: https://pretalx.com/fossgis2019/talk/EFGTTB/

\noindent%
\newTimeslot{18:00}
\abstractMathe{%
  Dominik Helle%
}{%
  Mitgliederversammlung FOSSGIS e.V.%
}{%
}{%
  Jährlich stattfindende Versammlung des FOSSGIS e.V. Alle Mitglieder sind herzlich eingeladen,
  teilzunehmen und sich zu beteiligen. Einige Themen stehen auf der Agenda. Wir laden ein zum
  Kennenlernen, zur Diskussion, Abstimmung \& Neuwahlen.%
}


%%%%%%%%%%%%%%%%%%%%%%%%%%%%%%%%%%%%%%%%%%%
