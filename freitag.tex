
% time: 09:00
\noindent%
\newTimeslot{09:00}
\abstractAudimax{%
  Tobias Frechen%
}{%
  Cloudbasierte Geodateninfrastruktur für den Glasfaserrollout in der Deutschen Telekom AG%
}{%
}{%
  Überblick über den Planungsprozess der Deutschen Telekom AG für den Glasfaserrollout mit Fokus auf den Aufbau und Prozesse einer cloudbasierten GDI.%
}


%%%%%%%%%%%%%%%%%%%%%%%%%%%%%%%%%%%%%%%%%%%

% time: 09:00
\noindent%

\abstractMathe{%
  Hans-Jörg Stark%
}{%
  Präsentation Ergebnisse Umfrage QGIS Nutzung in der Schweiz%
}{%
}{%
  Der Beitrag stellt die Ergebnisse der QGIS Anwendergruppevor, die im Herbst 2018 in der Schweiz durchgeführt wurde%
}


%%%%%%%%%%%%%%%%%%%%%%%%%%%%%%%%%%%%%%%%%%%

% time: 09:00
\noindent%

\abstractPhysik{%
  Daniel Koch, Marc Jansen%
}{%
  GeoServer: Live Demo-Session%
}{%
}{%
  In der Demo-Session werden wesentliche Funktionalitöten des GeoServers live demonstriert. Dies umfasst die grafische Benutzeroberfläche (Daten einbinden, FeatureTyoes erzeugen, Layer publizieren und stylen), die REST-Schnittstelle und ggf. zusätzliche Extensions.%
}


%%%%%%%%%%%%%%%%%%%%%%%%%%%%%%%%%%%%%%%%%%%

% time: 09:30
\noindent%
\newTimeslot{09:30}
\abstractAudimax{%
  Arnulf Christl%
}{%
  DevOps für die GDI 4.0: Agil Stabil.%
}{%
}{%
  Neue Trends und Technologien für die GDI 4.0. Ein Erfahrungsbericht aus dem Einsatz von Open Source Geodaten und OpenStreerMap in der Industrie.%
}


%%%%%%%%%%%%%%%%%%%%%%%%%%%%%%%%%%%%%%%%%%%

% time: 09:30
\noindent%

\abstractMathe{%
  Arndt Brenschede%
}{%
  Jenseits des etablierten OSM Routing-Regelsatzes%
}{%
}{%
  Etablierte Routing-Regeln entstehen nicht alleine durch Tagging-Proposals, sondern durch komplexe Wechselwirkungen, und in vielen Randbereichen auch garnicht. Ein Überblick.%
}


%%%%%%%%%%%%%%%%%%%%%%%%%%%%%%%%%%%%%%%%%%%

% time: 09:30
\noindent%

\abstractPhysik{%
  Alexey Valikov%
}{%
  Von statischen Bildern zu interaktiven Karten und zurück%
}{%
}{%
  Dieser Vortrag erläutert Werkzeuge und Techniken, womit man statische Bilder oder PDFs schnell und einfach in interaktive Karten umwandeln kann.%
}


%%%%%%%%%%%%%%%%%%%%%%%%%%%%%%%%%%%%%%%%%%%

% time: 10:00
\noindent%
\newTimeslot{10:00}
\abstractAudimax{%
  Arne Schubert%
}{%
  GDIs in der Cloud%
}{%
}{%
  Micro-Service-Infrastrukturen in einer Cloud bringen viele Vorteile doch auch
einige Nachteile gegenüber Monolithen mit sich. In dem Vortrag soll aufgezeigt
werden, was zu beachten ist, damit eine Migration oder die neue Infrastruktur in
der Cloud einen tatsächlichen Mehrwert bringt. Sei es was architektonisch zu
beachten ist, um die Vorteile für sich nutzen zu können oder welche
Möglichkeiten es gibt, um nicht von den Nachteilen betroffen zu sein.%
}


%%%%%%%%%%%%%%%%%%%%%%%%%%%%%%%%%%%%%%%%%%%

% time: 10:00
\noindent%

\abstractMathe{%
  Christoph Hormann%
}{%
  Wenn Mapper Karten malen%
}{%
}{%
  Warum bei der Datenerfassung in OpenStreetMap manchmal weniger mehr ist und was man als Mapper im Interesse von Erfassungs-Effizienz, der Daten-Qualität und einer möglichst breiten Nützlichkeit der Daten beachten sollte.%
}


%%%%%%%%%%%%%%%%%%%%%%%%%%%%%%%%%%%%%%%%%%%

% time: 10:00
\noindent%

\abstractPhysik{%
  Jakob Miksch%
}{%
  Malawi Atlas - Eine SDI mit PostGIS, GeoServer und GeoExt%
}{%
}{%
  Der Malawi Atlas ist eine Plattform um Naturgefahren in Malawi mittels Geodaten zu visualisieren. Im Hintergrund wird auf bewährte OpenSource Komponenten wie PostGIS, GeoServer, GeoExt und OpenLayers gesetzt.%
}


%%%%%%%%%%%%%%%%%%%%%%%%%%%%%%%%%%%%%%%%%%%

% time: 11:00
\noindent%
\newTimeslot{11:00}
\abstractAudimax{%
  Marc Jansen%
}{%
  GDI mit Docker \& Co. – Einführung, Überblick und Diskussion%
}{%
}{%
  Der talk stellt Möglichkeiten vor, um Geodateninfrastrukturen mit Hilfe von Docker zu gestalten und diskutiert jene.%
}


%%%%%%%%%%%%%%%%%%%%%%%%%%%%%%%%%%%%%%%%%%%

% time: 11:00
\noindent%

\abstractMathe{%
  Harald Schwarz%
}{%
  Erfassung der Düsseldorfer Gasbeleuchtung%
}{%
}{%
  Ich möchte auf der FOSSGIS-Konferenz 2019 mein Projekt Erfassung der Düsseldorfer Gasbeleuchtung in OpenStreetMap vorstellen.  




Ich habe seit 2010 den kompletten Bestand der Düsseldorfer Gasbeleuchtung
(ca. 15000 Gaslaternen, 1200 gasbeleuchtete Straßen)
in Open\textbackslash-Street\textbackslash-Map erfasst.

Ich möchte berichten, welche Erfahrungen ich bei meinen Spaziergängen durch Düsseldorf mit GPS und Fotokamera gemacht habe, wie die Daten in OSM eingepflegt wurden und
wie die OSM Daten nutzbar gemacht w%
}


%%%%%%%%%%%%%%%%%%%%%%%%%%%%%%%%%%%%%%%%%%%

% time: 11:00
\noindent%

\abstractPhysik{%
  Karsten Vennemann, Pirmin Kalberer%
}{%
  Vergleich und Benchmark der Generierung von Karten-Vektorkacheln via MapServer versus t-rex%
}{%
}{%
  Karten-Vektorkacheln (aka vector tiles) sind ein Datenformat das besonders für interaktive Web GIS Anwendungen interessant ist. Die Präsentation stellt zwei technische Möglichkeiten Vector Tiles zu erstellen vor. Dabei wird MapServer als traditioneller Map Rendering Engine wird mit dem neuen Paket T-REX, einem Vector Tile Server in der Konfiguration und in einem Benchmark Test verglichen.%
}


%%%%%%%%%%%%%%%%%%%%%%%%%%%%%%%%%%%%%%%%%%%

% time: 11:30
\noindent%
\newTimeslot{11:30}
\abstractAudimax{%
  Volker Mische%
}{%
  Einführung in dezentrale Infrastrukturen und IPFS%
}{%
}{%
  Werden Geodaten auf einem verteilten, dezentralen System gespeichert bietet dies einige Vorteile. Es führt zu größerer Ausfallsicherheit, besser Erreichbarkeit und mehr Sicherheit. Diese Vortrag gibt einen Einblick in diese Entwicklung und zeigt anhand von IPFS, dem [InterPlanetary Filesystem], wie dies dann im Alltag aussieht und welche Vorteile es bietet. IPFS ist Open Source unter [MIT Lizenz].

[InterPlanetary Filesystem]: https://ipfs.io/
[MIT Lizenz]: https://opensource.org/licenses/MIT%
}


%%%%%%%%%%%%%%%%%%%%%%%%%%%%%%%%%%%%%%%%%%%

% time: 11:30
\noindent%

\abstractMathe{%
  Sarah Hoffmann%
}{%
  Im Frühtau zu Berge - 10 Jahre Wanderkarten mit OSM%
}{%
}{%
  Zum 10-jährigen Bestehen der Wanderkarte waymarkedtrails.org, blickt der Vortrag auf deren Entwicklung zurück und stellt vor, was sich beim Mapping für Wander-, Rad- und andere Routen in OpenStreetMap getan hat.%
}


%%%%%%%%%%%%%%%%%%%%%%%%%%%%%%%%%%%%%%%%%%%

% time: 11:30
\noindent%

\abstractPhysik{%
  Thomas Skowron%
}{%
  Vector Tiles hinter den Kulissen%
}{%
}{%
  Vector Tiles verdrängen an vielen Stellen Bitmaps, aber wie werden sie gemacht? Wieso setzt sich das MVT Format durch und was kann es? …und wurde WMS neu erfunden?%
}


%%%%%%%%%%%%%%%%%%%%%%%%%%%%%%%%%%%%%%%%%%%

% time: 12:00
\noindent%
\newTimeslot{12:00}
\abstractAudimax{%
  Christian Strobl%
}{%
  Das "`Cloud Optimized GeoTIFF"' - wenig Theorie und viel Praxis%
}{%
}{%
  Ein "`Cloud Optimized GeoTIFF"' ist eine normale GeoTIFF-Datei, die eine spezielle interne Struktur aufweist, die für spezielle HTTP-Aufrufe optimiert ist. Neben einer Vorstellung der Spezifikation und typischen Anwendungsbeispielen wird die Erstellung von "`Cloud Optimized GeoTIFFs"' mit GDAL vorgestellt. Dabei werden unterschiedliche Strategien zur Erstellung von Overviews diskutiert, die die Verarbeitungszeit erheblich verkleinen können.%
}


%%%%%%%%%%%%%%%%%%%%%%%%%%%%%%%%%%%%%%%%%%%

% time: 12:00
\noindent%

\abstractMathe{%
  Alexander Matheisen%
}{%
  Produktion generalisierter Eisenbahnkarten%
}{%
}{%
  In diesem Vortrag wird die Erstellung von hochwertigen, Topologie-fähigen Eisenbahn-Streckenkarten auf der Basis von OpenStreetMap-Daten vorgestellt. Der verwendete Workflow kombiniert automatisierte Prozesse mit manueller Bearbeitung und nutzt moderne, datengetriebene Vektor-Tile-Technologien für die Publikation als Webkarten.%
}


%%%%%%%%%%%%%%%%%%%%%%%%%%%%%%%%%%%%%%%%%%%

% time: 12:00
\noindent%

\abstractPhysik{%
  Alexey Valikov%
}{%
  Offene Fahrplandaten in Deutschland%
}{%
}{%
  In diesem Vortrag berichten wir über den Stand der offenen Fahplandaten in Deutschland und Nachbarländern.%
}


%%%%%%%%%%%%%%%%%%%%%%%%%%%%%%%%%%%%%%%%%%%

% time: 12:05
\noindent%
\newTimeslot{12:05}
\abstractPhysik{%
  Jörg Höttges, Christian Lassert%
}{%
  Ein interaktiver Hochwassereinsatzplan out of the USB-Stick%
}{%
}{%
  Mit QGIS wurde ein mobiler und interaktiver Hochwassereinsatzplan erstellt, der für die Gewässerunterhaltung eines Wasserverbandes alle wichtigen Informationen offline bereitstellt, aber auch Online-Daten verwenden kann. Das System ist sehr einfach zu installieren, da der gesamte Datenbestand inklusive aller notwendigen Installationsdateien auf einem USB-Stick gespeichert ist.%
}


%%%%%%%%%%%%%%%%%%%%%%%%%%%%%%%%%%%%%%%%%%%

% time: 12:10
\noindent%
\newTimeslot{12:10}
\abstractPhysik{%
  Jakob Miksch%
}{%
  Maptime Salzburg - Ein Meetup für Geo-Themen%
}{%
}{%
  Wir haben das Meetup namens "`Maptime Salzburg"' gegründet. Der Vortrag berichtet über unsere Erfahrungen.%
}


%%%%%%%%%%%%%%%%%%%%%%%%%%%%%%%%%%%%%%%%%%%

% time: 13:30
\noindent%
\newTimeslot{13:30}
\abstractAudimax{%
  Anita Graser%
}{%
  Keynote: Einblicke vom Bazaar des QGIS-Projekts%
}{%
}{%
  Ein Blick hinter die Kulissen des QGIS-Projekts: Woher es kommt, wie es tickt und wohin es weiter geht%
}


%%%%%%%%%%%%%%%%%%%%%%%%%%%%%%%%%%%%%%%%%%%

% time: 14:00
\noindent%
\newTimeslot{14:00}
\abstractAudimax{%
  Dominik Helle, Frank Schwarzbach%
}{%
  Abschluss der drei Konferenztage. Es geht nach dem Sektempfang mit OSM weiter%
}{%
}{%
  Abschluss der drei Konferenztage. Es geht nach dem Sektempfang mit OSM weiter - mit dem berühmten OSM-Quiz, Exkursion, Diskussion, Vorträgen...%
}


%%%%%%%%%%%%%%%%%%%%%%%%%%%%%%%%%%%%%%%%%%%

% time: 14:30
\noindent%
\newTimeslot{14:30}
\abstractOther{%
  Astrid Emde%
}{%
  Sektempfang findet im PAB im Ausstellerbereich statt.%
}{%
}{%
  Der FOSSGIS-Verein lädt seine Mitglieder, Freunde, Unterstützer und Interessierte herzlich zum geselligen Beisammensein nach Abschluss der Konferenz an den FOSSGIS-Stand sein.%
}


%%%%%%%%%%%%%%%%%%%%%%%%%%%%%%%%%%%%%%%%%%%

% time: 15:00
\noindent%
\newTimeslot{15:00}
\abstractAudimax{%
  Christopher Lorenz%
}{%
  OSM-Quiz - wie gut kennst du OSM?%
}{%
}{%
  Das OSM-Quiz bietet als Fortsetzung der Events der letzten Jahre wieder spannende Fragen zu interessanten Fakten. Jeder ist herzlich eingeladen mitzuraten um sein Wissen im Umfeld von OpenStreetMap und GIS zu testen und vielleicht auch etwas aufzufrischen.%
}


%%%%%%%%%%%%%%%%%%%%%%%%%%%%%%%%%%%%%%%%%%%

% time: 15:30
\noindent%
\newTimeslot{15:30}
\abstractAudimax{%
  Marc Tobias Metten%
}{%
  Küsten, Meere, Zeitzonen  - uneditierbar grosse Polygone%
}{%
}{%
  Wir schauen auf mehrere Datensätze, die auf OpenStreetMap basieren aber aufgrund ihrer Größe ohne separate Prozesse nicht in den üblichen Editoren bearbeitet werden können. Als Beispiel erstellen wir ein Polygon der Eurostaaten.%
}


%%%%%%%%%%%%%%%%%%%%%%%%%%%%%%%%%%%%%%%%%%%

% time: 15:30
\noindent%

\abstractMathe{%
  Hartmut Holzgraefe%
}{%
  Viele Kartenstile parallel installieren%
}{%
}{%
  Verschiedene Mapnik-Kartenstile haben unterschiedliche Datenbank- ud Shapefile-Abhängigkeiten. Trotzdem ist es mit ein klein wenig Aufwand möglich die meisten öffentlich verfügbaren Stile parallel zu installieren.%
}


%%%%%%%%%%%%%%%%%%%%%%%%%%%%%%%%%%%%%%%%%%%

% time: 15:30
\noindent%

\abstractOther{%
  Holger%
}{%
  Exkursion zum Mathematisch-Physikalischen Salon im Dresdener Zwinger%
}{%
}{%
  Exkursion zum Mathematisch-Physikalischen Salon. Der Salon ist ein Museum im Zwinger https://mathematisch-physikalischer-salon.skd.museum/
Treffpunkt: 15:30 Uhr im Eingangsbereich der HTW%
}


%%%%%%%%%%%%%%%%%%%%%%%%%%%%%%%%%%%%%%%%%%%

% time: 16:30
\noindent%
\newTimeslot{16:30}
\abstractAudimax{%
  Falk Zscheile%
}{%
  Craftmapping und Datenschutzgrundverordnung - Was ist erlaubt,   wo sind die Grenzen?%
}{%
}{%
  Das OpenStreetMap-Projekt sammelt geographische Informationen. Dabei ergeben sich zahlreiche Berührungspunkte mit personenbezogenen Informationen. Der Vortrag geht den Fragen nach, ob die  Datenschutzgrundverordnung einschlägig ist und welche Anforderungen  ggf. beim Craftmapping und der Eintragung von Informationen in die Datenbank zu beachten sind.%
}


%%%%%%%%%%%%%%%%%%%%%%%%%%%%%%%%%%%%%%%%%%%

% time: 16:30
\noindent%

\abstractMathe{%
  Hanna Krüger%
}{%
  Das OSM-Wiki - Die eierlegende Wollmilchsau der Community%
}{%
}{%
  Ein Rundumschlag zum OSM-Wiki: Was steht eigentlich alles drin, welche Stärken und Schwächen hat das Konzept und wie könnte man den Problemen des Wikis entgegenwirken%
}


%%%%%%%%%%%%%%%%%%%%%%%%%%%%%%%%%%%%%%%%%%%

% time: 17:00
\noindent%
\newTimeslot{17:00}
\abstractMathe{%
  Christopher Lorenz%
}{%
  Fahrradknotenpunkte in OpenStreetMap%
}{%
}{%
  Der Vortrag zeigt, was Fahrradknotenpunkte sind und wie sie in OpenStreetMap erfasst werden. Neben der Darstellung der länderabhängigen Verbreitung und Erfassung der Knotenpunkte wird die Machbarkeit einer Auswertung der Knotenpunkte in OpenStreetMap präsentiert.%
}


%%%%%%%%%%%%%%%%%%%%%%%%%%%%%%%%%%%%%%%%%%%
