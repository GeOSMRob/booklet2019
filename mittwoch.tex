
% time: Wednesday 10:30
% URL: https://pretalx.com/fossgis2019/talk/JESWEE/

\noindent%
\newTimeslot{10:30}
\abstractMathe{%
  Dominik Helle%
}{%
  Was sind "`Open"' Source, Data und Standards und wie funktioniert das?%
}{%
}{%
  Der Vortrag stellt die Geschichte der Entwicklung von Open Source vor und geht auf wichtige
  Grundlagen ein


  Ziel des FOSSGIS e.V. und der OSGeo ist die Förderung und Verbreitung freier Geographischer
  Informationssysteme (GIS) im Sinne Freier Software und Freier Geodaten. Dazu zählen auch
  Erstinformation und Klarstellung von typischen Fehlinformationen über Open Source und Freie
  Software, die sich über die Jahre festgesetzt haben.%
}


%%%%%%%%%%%%%%%%%%%%%%%%%%%%%%%%%%%%%%%%%%%

% time: Wednesday 11:00
% URL: https://pretalx.com/fossgis2019/talk/UHJECS/

\noindent%
\newTimeslot{11:00}
\abstractMathe{%
  Felix Kunde%
}{%
  Tour de FOSS4G~-- Eine Reise durch den großen Dschungel freier Software für Geodaten%
}{%
}{%
  Es gibt so viele tolle Open-Source-Projekte im FOSS4G-Umfeld, von denen man als FOSSGIS-Besucher
  eventuell nichts mitbekommt. Entweder kommen die Kernentwickler nicht aus Deutschland oder sie
  setzen mal ein Jahr aus oder sie mögen einfach nicht vortragen etc.%
}


%%%%%%%%%%%%%%%%%%%%%%%%%%%%%%%%%%%%%%%%%%%

% time: Wednesday 11:30
% URL: https://pretalx.com/fossgis2019/talk/SVZSSA/

\noindent%
\newTimeslot{11:30}
\abstractMathe{%
  Wolfgang Hinsch%
}{%
  Einführung in OpenStreetMap%
}{%
}{%
  Die Entstehung, die heutige Bedeutung und die Vielseitigkeit von OpenStreetMap werden vorgestellt.
  Es werden Möglichkeiten zur Mitwirkung aufgezeigt.%
}


%%%%%%%%%%%%%%%%%%%%%%%%%%%%%%%%%%%%%%%%%%%

% time: Wednesday 13:00
% URL: https://pretalx.com/fossgis2019/talk/ASZ3EL/

\noindent%
\newTimeslot{13:00}
\abstractAudimax{%
  Dominik Helle, Frank Schwarzbach, Dr. Frank Pfeil%
}{%
  Eröffnung der Konferenz 2019%
}{%
}{%
  Eine feierliche Eröffnung der Konferenz durch Vertreter des FOSSGIS e.V. und der HTW Dresden mit
  wertvollen Hinweisen zum Ablauf und der Organisation.%
}


%%%%%%%%%%%%%%%%%%%%%%%%%%%%%%%%%%%%%%%%%%%

% time: Wednesday 13:30
% URL: https://pretalx.com/fossgis2019/talk/RD33ZJ/

\noindent%
\newTimeslot{13:30}
\abstractAudimax{%
  Johannes Terwyen%
}{%
  OSM und öffentliche Verwaltung~-- Wie geht das?%
}{%
}{%
  Der Regionalverband Ruhr und seine Partner entwickeln zurzeit eine neue Stadtkarte auf der Basis
  von ALKIS- und OSM-Daten. Nach einer kleinen Einführung in die Inhalte und die Technik beleuchtet
  dieser Beitrag das Projekt insbesondere unter dem Aspekt der "`Zusammenarbeit"' der Kommunen und der
  OSM-Community. Beschrieben wird der Prozess des "`Kennenlernens"' und "`Verstehens"'. Im Kern geht es
  um Respekt, Akzeptanz und Toleranz als Basis für eine fruchtbare Zusammenarbeit zum beiderseitigen
  Vorteil.%
}


%%%%%%%%%%%%%%%%%%%%%%%%%%%%%%%%%%%%%%%%%%%

% time: Wednesday 14:05
% URL: https://pretalx.com/fossgis2019/talk/ZRFTV3/

\noindent%
\newTimeslot{14:05}
\abstractAudimax{%
  Arnulf Christl%
}{%
  Über die Motivation von KI%
}{%
}{%
  "`Artifical Intelligence"', zu Deutsch "`Künstliche Intelligenz"' ist ein großer Begriff, der mit
  vielen Ängsten, Wünschen und sonstigen Projektionen überladen ist. Lernende Algorithmen dagegen
  hört sich harmlos an. Sind sie aber nicht, meint ein Ex-Borg und möchte das in einem Lightning
  Talk erläutern.%
}


%%%%%%%%%%%%%%%%%%%%%%%%%%%%%%%%%%%%%%%%%%%

% time: Wednesday 14:10
% URL: https://pretalx.com/fossgis2019/talk/DRPBMB/

\noindent%
\newTimeslot{14:10}
\abstractAudimax{%
  Christian Mayer%
}{%
  Was sollen diese komischen Adress-Codes?%
}{%
}{%
  Geografische Positionen werden klassischer Weise in (projizierten) Koordinaten angegeben.
  Mittlerweile gibt es zu diesen Koordinatenangaben mehrere alternative Adressierungssysteme.  Sei
  es das offene "`Open Location Code"', "`Mapcode"' von einer niederländische Non-Profit-Organisation,
  das proprietäre "`what3words"' oder die (nicht ganz erst gemeinten) "`what3ducks"' und "`what3ikea"

  Wer nutzt solche Adresssysteme und braucht die Welt diese überhaupt?%
}


%%%%%%%%%%%%%%%%%%%%%%%%%%%%%%%%%%%%%%%%%%%

% time: Wednesday 14:15
% URL: https://pretalx.com/fossgis2019/talk/CLXMN7/

\noindent%
\newTimeslot{14:15}
\abstractAudimax{%
  Astrid Emde%
}{%
  OSGeo~-- Teil einer weltweiten Community%
}{%
}{%
  FOSSGIS vertritt D-A-C-H. Doch in Wirklichkeit ist FOSSGIS Teil einer globalen Community. Eine
  Community, die sehr aktiv ist. OSGeo, OSM, FOSS4G finden weltweit und ständig statt.%
}


%%%%%%%%%%%%%%%%%%%%%%%%%%%%%%%%%%%%%%%%%%%

% time: Wednesday 15:00
% URL: https://pretalx.com/fossgis2019/talk/DZSY8M/

\noindent%
\newTimeslot{15:00}
\abstractAudimax{%
  Pirmin Kalberer%
}{%
  Routing mit Open-Source-Software%
}{%
}{%
  pgRouting kennen die Meisten, aber für viele Anwendungsfälle sind andere Open-Source-Tools
  mindestens ebenso gut geeignet. Dieser Vortrag zeigt Anwendungen vom Eisenbahn- bis zum
  Schiffsrouting und gibt Tipps zum Einsatz von geeigneten Routing-Technologien.%
}


%%%%%%%%%%%%%%%%%%%%%%%%%%%%%%%%%%%%%%%%%%%

% time: Wednesday 15:00
% URL: https://pretalx.com/fossgis2019/talk/H3WKUA/

\noindent%

\abstractMathe{%
  Christian Clemen%
}{%
  BIM- und GIS-Interoperabilität%
}{%
}{%
  Der Vortrag vergleicht die Methoden der Informationsverarbeitung der Geo- und Bauwelt und stellt
  die Zwischenergebnisse einer gemeinsamen Arbeitsgruppe (ISO TC211 und ISO TC59 SC13) "`BIM/GIS
  Interoperability"' vor.%
}


%%%%%%%%%%%%%%%%%%%%%%%%%%%%%%%%%%%%%%%%%%%

% time: Wednesday 15:00
% URL: https://pretalx.com/fossgis2019/talk/QW7UCE/

\noindent%

\abstractPhysik{%
  Niklas Alt%
}{%
  FOSSGIS im Museum~-- eine digitale historische Sozialtopographie%
}{%
}{%
  Der Vortrag stellt eine komplexe Geo-Anwendung (OpenLayers + Angular6) vor, die für die
  Karl-Marx-Landesausstellung in Trier entwickelt wurde und im dortigen Landesmuseum auf einem
  großformatigen Bildschirm den Besucher einen Einblick in die Unterschicht Triers zur Jugendzeit
  von Karl Marx bot. Neben einer kurzen Einführung des historischen Kontext und die technische
  Umsetzung wird auch das Nutzungsverhalten thematisiert.%
}


%%%%%%%%%%%%%%%%%%%%%%%%%%%%%%%%%%%%%%%%%%%

% time: Wednesday 15:30
% URL: https://pretalx.com/fossgis2019/talk/LLMRCA/

\noindent%
\newTimeslot{15:30}
\abstractAudimax{%
  Peter%
}{%
  GraphHopper-Routing-Engine~-- Einblicke und Ausblick%
}{%
}{%
  Der Vortrag wird Einblicke in vergangene und aktuelle Entwicklungen liefern. Auch wird ein
  Ausblick auf kommende Features nicht fehlen.%
}


%%%%%%%%%%%%%%%%%%%%%%%%%%%%%%%%%%%%%%%%%%%

% time: Wednesday 15:30
% URL: https://pretalx.com/fossgis2019/talk/FEJJ3L/

\noindent%

\abstractMathe{%
  Yaseen Srewil%
}{%
  OpenBIM zur Unterstützung der Wohnungswirtschaft basierend auf einer PostGIS-Datenbank und BIMServer.org%
}{%
}{%
  Die Kombination von BIM-Modellen und Geodaten ist eine Schlüsselfunktion für ein einheitliches
  digitales Abbild der gebauten Umwelt. Im Rahmen des BMWi-Projektes "`IMMOMATIK"' soll die
  BIM-Methode auf den Betrieb von Immobilien der Wohnungswirtschaft angewendet werden. Der Fokus
  liegt auf den Daten, die ausschließlich mit Open-Source-Software im BIMServer.org und in einer
  PostGIS Datenbank geführt werden. So können die Vorteile der OpenBIM-Methodik und Geodaten genutzt
  werden.%
}


%%%%%%%%%%%%%%%%%%%%%%%%%%%%%%%%%%%%%%%%%%%

% time: Wednesday 15:30
% URL: https://pretalx.com/fossgis2019/talk/MNS38N/

\noindent%

\abstractPhysik{%
  Lina Dillmann%
}{%
  Usability Testing in GIS%
}{%
}{%
  Bedien- und Benutzerfreundlichkeit in Software kann effektiv getestet werden. Welche Testmethoden
  es gibt und was dabei zu beachten ist, wird im Vortrag erklärt.%
}


%%%%%%%%%%%%%%%%%%%%%%%%%%%%%%%%%%%%%%%%%%%

% time: Wednesday 16:00
% URL: https://pretalx.com/fossgis2019/talk/NVK9MY/

\noindent%
\newTimeslot{16:00}
\abstractAudimax{%
  Bernd Marcus%
}{%
  Abseits der öffentlichen Straßen~-- Eine Routenplanung auf OSM-Basis mit SpatiaLite und QGIS%
}{%
}{%
  Datensätze kommerzieller Navigationssysteme decken Wegenetze außerhalb des öffentlichen
  Straßennetzes meist nur unzureichend ab. OSM-Daten können diese Lücke in vielen Fällen erfolgreich
  schließen. Am Beispiel zum Auffinden forstlicher Holzlagerplätze werden die Aspekte zum Aufbau von
  Routen fähigen Straßendaten erläutert und eine QGIS-Anwendung mit SpatiaLite Unterbau vorgestellt,
  mit der sich aufgrund fehlender Hausadressierung im Wald die Navigationsstrecke mittels Maus
  zusammenstellen lässt.%
}


%%%%%%%%%%%%%%%%%%%%%%%%%%%%%%%%%%%%%%%%%%%

% time: Wednesday 16:00
% URL: https://pretalx.com/fossgis2019/talk/B3CYF3/

\noindent%

\abstractMathe{%
  Steffen Hollah, Martin Dresen%
}{%
  Building Information Modeling mit Open-Source-Tools%
}{%
}{%
  Mit den Open-Source-Tools OpenLayers und Cesium lassen sich Gebäudemodelle in 2D und 3D
  nebeneinander visualisieren. Das Framework ol-cesium ermöglicht dabei eine einfache
  Synchronisierung zwischen 2D- und 3D-Ansichten. Im Vortrag werden verschiedene Lösungen und
  Beispiele gezeigt, die eine Visualisierung von Gebäudemodellen in Kombination mit Geodiensten und
  Hintergrundkarten ermöglichen.%
}


%%%%%%%%%%%%%%%%%%%%%%%%%%%%%%%%%%%%%%%%%%%

% time: Wednesday 16:00
% URL: https://pretalx.com/fossgis2019/talk/8FHRP9/

\noindent%

\abstractPhysik{%
  Christin Henzen, Lisa Eichler%
}{%
  Ich sehe was, was Du nicht siehst~-- die Bewertung der Usability freier Web-GIS am Beispiel einer Eyetracking-Studie zum IÖR-Monitor%
}{%
}{%
  Die Usability frei zugänglicher Web-GIS variiert derzeit stark. In einer Usability-Studie wurden
  Usability-Probleme und Verbesserungspotenziale freier Web-GIS am Beispiel des IÖR-Monitors im
  Spannungsfeld zwischen subjektiven Nutzerbewertungen und objektiv gemessenen Eyetrackingdaten
  ermittelt. Die so entstandene Sammlung von Problemen und Lösungsvorschlägen kann während der
  Entwicklung oder Überarbeitung von Web-GIS oder zugrundeliegender APIs zur Verbesserung der
  Usability genutzt werden.%
}


%%%%%%%%%%%%%%%%%%%%%%%%%%%%%%%%%%%%%%%%%%%

% time: Wednesday 17:00
% URL: https://pretalx.com/fossgis2019/talk/JWSVDN/

\noindent%
\newTimeslot{17:00}
\abstractAudimax{%
  Max Bohnet, Christoph Franke%
}{%
  QGIS-Plugins zum Geocoding und zu intermodaler Erreichbarkeitsanalyse mit dem OpenTripPlanner%
}{%
}{%
  Zwei QGIS-PlugIns werden vorgestellt, die die Funktionalitäten von Geokodierdiensten und von
  intermodalen Erreichbarkeitsanalysen mit dem OpenTripPlanner in das Desktop-GIS integrieren.%
}


%%%%%%%%%%%%%%%%%%%%%%%%%%%%%%%%%%%%%%%%%%%

% time: Wednesday 17:00
% URL: https://pretalx.com/fossgis2019/talk/AXRZKV/

\noindent%

\abstractMathe{%
  Dirk Stenger%
}{%
  TEAM Engine~-- Validierung des neuen OGC-Standards WFS 3.0 und aktuelle Entwicklungen im Projekt%
}{%
}{%
  Die TEAM Engine ist eine Engine, mit der Entwickler und Anwender Geodienste, wie WFS und WMS, und
  Geoformate, wie GML oder GeoPackage, testen können


  Dieser Vortrag stellt vor, wie der neue OGC-Standard WFS 3.0 mit der TEAM Engine validiert werden
  kann. Dabei wird der Prozess der Erstellung der neuen Testsuite im Rahmen des
  OGC-Testbed-14-Programms näher beleuchtet. Des Weiteren werden die aktuellen Entwicklungen im
  TEAM-Engine-Projekt aufgezeigt und ein Ausblick gegeben.%
}


%%%%%%%%%%%%%%%%%%%%%%%%%%%%%%%%%%%%%%%%%%%

% time: Wednesday 17:00
% URL: https://pretalx.com/fossgis2019/talk/7WMTLR/

\noindent%

\abstractPhysik{%
  Tim Alder%
}{%
  Pointclouds für OSM%
}{%
}{%
  Der Vortrag beschreibt ein selbstgebautes Geräte zur Laserscandatenerfassung und anschließende
  Visualisierung mittels Potree.%
}


%%%%%%%%%%%%%%%%%%%%%%%%%%%%%%%%%%%%%%%%%%%

% time: Wednesday 17:05
% URL: https://pretalx.com/fossgis2019/talk/MNKP8V/

\noindent%
\newTimeslot{17:05}
\abstractPhysik{%
  Johannes Kröger%
}{%
  Leider kein LiDAR?%
}{%
}{%
  Ich erzähle kurz zum Stand der Dinge in meinen Versuchen amtliche LiDAR-Daten bzw. das digitale
  Oberflächenmodell der Stadt Hamburg über das Hamburgische Transparenzgesetz zu befreien.%
}


%%%%%%%%%%%%%%%%%%%%%%%%%%%%%%%%%%%%%%%%%%%

% time: Wednesday 17:10
% URL: https://pretalx.com/fossgis2019/talk/HHM9GT/

\noindent%
\newTimeslot{17:10}
\abstractPhysik{%
  Felix Kunde%
}{%
  GPU Datenbanken%
}{%
}{%
  Was sind GPU-Datenbanken? Wie viel Geo steckt schon drin? Eine kurze Live-Demo mit OmniSci wirds
  euch zeigen.%
}


%%%%%%%%%%%%%%%%%%%%%%%%%%%%%%%%%%%%%%%%%%%

% time: Wednesday 17:15
% URL: https://pretalx.com/fossgis2019/talk/SKCRTV/

\noindent%
\newTimeslot{17:15}
\abstractPhysik{%
  Alexey Valikov%
}{%
  Preis der Karte%
}{%
}{%
  Wie sehen die Preise der kommerziellen Tile Servern aus

  In diesem Lightning Talk teilen wir die Ergebnisse unserer Analyze des Kartendienst-Marktes.%
}


%%%%%%%%%%%%%%%%%%%%%%%%%%%%%%%%%%%%%%%%%%%

% time: Wednesday 17:30
% URL: https://pretalx.com/fossgis2019/talk/DTHTXK/

\noindent%
\newTimeslot{17:30}
\abstractAudimax{%
  Holger Bruch%
}{%
  Mitfahren-BW –ÖPNV und Fahrgemeinschaften intermodal mit dem OpenTripPlanner%
}{%
}{%
  Freie, intermodale Open-Source-Routenplaner wie OpenTripPlanner, zunehmend als Open Data
  veröffentlichte ÖPNV-Fahrpläne sowie OpenStreetMap machen neue, innovative Anwendungen möglich, um
  z.B. Pendlern Alternativen zur Fahrt im eigenen Auto anzubieten.%
}


%%%%%%%%%%%%%%%%%%%%%%%%%%%%%%%%%%%%%%%%%%%

% time: Wednesday 17:30
% URL: https://pretalx.com/fossgis2019/talk/PRYYVA/

\noindent%

\abstractMathe{%
  Matthias Kuhn%
}{%
  QGIS-Projektgenerator~-- vom Datenmodell zur Erfassung%
}{%
}{%
  Das Plugin QGIS-Projektgenerator wird vorgestellt. Dieses erlaubt es, aus PostGIS, GeoPackage oder
  Interlis-Datenmodellen ansprechende Erfassungsmasken zu erstellen


  Dabei wird insbesondere auch auf die Herausforderungen eingegangen, die sich ergeben, wenn Daten
  nach einheitlichem Schema von verschiedenen Stellen erfasst werden sollen.%
}


%%%%%%%%%%%%%%%%%%%%%%%%%%%%%%%%%%%%%%%%%%%

% time: Wednesday 17:30
% URL: https://pretalx.com/fossgis2019/talk/E7TBRX/

\noindent%

\abstractPhysik{%
  Felix Kunde%
}{%
  Räumliche Indizes in PostGIS~-- Welcher ist der richtige?%
}{%
}{%
  GIST, sp-GIST, BRIN oder BTREE. In PostgreSQL gibt es verschiedene Indextypen, die mittlerweile
  auch für PostGIS-Geometriespalten unterstützt werden. Der Vortrag wird kurz die wesentlichen
  Unterschiede vorstellen und anhand von einfachen Regressionstests mit künstlichen Geodaten
  aufzeigen, wann welcher Typ am Besten verwendet werden sollte.%
}


%%%%%%%%%%%%%%%%%%%%%%%%%%%%%%%%%%%%%%%%%%%

% time: Wednesday 18:00
% URL: https://pretalx.com/fossgis2019/talk/DJNTBQ/

\noindent%
\newTimeslot{18:00}
\abstractAudimax{%
  Simon Nieland%
}{%
  Urban Mobility Accessibility Computer (UrMoAC)~-- ein Open Source Tool zur Berechnung von Erreichbarkeitsmaßen%
}{%
}{%
  Der Urban Mobility Accessibility Computer ist ein Open-Source-Tool zur Berechnung von urbanen
  Erreichbarkeitsmaßen. Es wird dazu verwendet Verkehrssysteme hinsichtlich ihrer Effektivität und
  Verfügbarkeit zu bewerten und stellt somit eine wertvolle Grundlage für städtische Verkehrsplanung
  dar.%
}


%%%%%%%%%%%%%%%%%%%%%%%%%%%%%%%%%%%%%%%%%%%

% time: Wednesday 18:00
% URL: https://pretalx.com/fossgis2019/talk/ESDMQB/

\noindent%

\abstractMathe{%
  Andreas Schmid%
}{%
  Geodatenmanagement mit GRETL%
}{%
}{%
  Beim Amt für Geoinformation des Kantons Solothurn steht seit 2018 das Datenmanagement-Tool GRETL
  im Einsatz für den Datenimport und -export in die bzw. aus der PostGIS-Datenbank, aber auch für
  Datenumbauten von einem Datenmodell in ein anderes. GRETL ist ein Plugin für das *Gradle Build
  Tool*, wodurch die volle Power eines Build-Tools neu für Geodatenflüsse zur Verfügung steht.%
}


%%%%%%%%%%%%%%%%%%%%%%%%%%%%%%%%%%%%%%%%%%%

% time: Wednesday 18:00
% URL: https://pretalx.com/fossgis2019/talk/WV3EPA/

\noindent%

\abstractPhysik{%
  Andreas Kretschmer%
}{%
  PostgreSQL: EXPLAIN erklärt%
}{%
}{%
  Bei der Analyse von Performance-Problemen (Warum ist diese Abfrage langsam) kann eine Auswertun

  des Abfrageplanes via EXPLAIN helfen. Doch wie liest man dies?%
}


%%%%%%%%%%%%%%%%%%%%%%%%%%%%%%%%%%%%%%%%%%%

% time: Wednesday 18:30
% URL: https://pretalx.com/fossgis2019/talk/FGESN7/

\noindent%
\newTimeslot{18:30}
\abstractMathe{%
  Arne Schubert, Stephan Herritsch%
}{%
  YAGA Anwendertreffen%
}{%
}{%
  Das YAGA Entwicklerteam bietet eine Vielzahl von OpenSource Projekten an, mit denen leicht moderne
  Web-, App- und Server-Anwendungen erstellt werden können. Darunter leaflet-ng2, einer granularen
  Integration von Leaflet in Angular.io, sowie diverse Docker-Images zum Aufbau einer
  Geodateninfrastuktur (GDI). Das Anwendertreffen bietet die Möglichkeit zum gegenseitigen
  Austausch, sowie Hilfestellungen bei Problemen und Fragen.%
}


%%%%%%%%%%%%%%%%%%%%%%%%%%%%%%%%%%%%%%%%%%%

% time: Wednesday 18:30
% URL: https://pretalx.com/fossgis2019/talk/CVCDKS/

\noindent%

\abstractPhysik{%
  Astrid Emde%
}{%
  PostNAS-Suite Anwendertreffen%
}{%
}{%
  Die PostNAS-Suite Anwender treffen sich jedes halbe Jahr zum Austausch. Das nächste Treffen soll
  auf der FOSSGIS 2019 stattfinden. Hier sollen aktuelle Entwicklungen im PostNAS-Suite Projekt
  vorgestellt werden. Das Treffen richtet sich besonders an Neueinsteiger, die die Möglichkeiten
  kennenlernen wollen.%
}


%%%%%%%%%%%%%%%%%%%%%%%%%%%%%%%%%%%%%%%%%%%

% time: Wednesday 19:00
% URL: https://pretalx.com/fossgis2019/talk/ALENWA/

\noindent%
\newTimeslot{19:00}
\abstractOther{%
  Dominik Helle%
}{%
  Dialoge im Bärenzwinger%
}{%
}{%
  Für die Abendveranstaltung "`Dialoge im Bärenzwinger"' wurde der Bärenzwinger Dresden ausgewählt.
  Für das Buffet steht außerdem das direkt daran angrenzende historische Kassemattengewölbe zur
  Verfügung.


  https://www.baerenzwinger.de


  Der Studentenclub liegt direkt an der historischen Altstadt, den Brühlschen Terrassen, und ist per
  Straßenbahnlinie 3, 7, 8 und 9 erreichbar. Zu Fuß ist die Veranstaltung ab Zentralgebäude etwa
  30min entlang der Einkaufsstraße "`Prager Straße"' entfernt.%
}


%%%%%%%%%%%%%%%%%%%%%%%%%%%%%%%%%%%%%%%%%%%
