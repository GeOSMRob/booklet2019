
% time: 10:30
\noindent%
\newTimeslot{10:30}
\abstractMathe{%
  Dominik Helle%
}{%
  Was sind "`Open"' Source, Data und Standards - und wie funktioniert das?%
}{%
}{%
  TEXT NOCH ANPASSEN ... Der Vortrag stellt die Geschichte der Entwicklung von Open Source vor und geht auf wichtige Grundlagen ein.
Ziel des FOSSGIS e.V. und der OSGeo ist die Förderung und Verbreitung freier Geographischer Informationssysteme (GIS) im Sinne Freier Software und Freier Geodaten. Dazu zählen auch Erstinformation und Klarstellung von typischen Fehlinformationen über Open Source und Freie Software, die sich über die Jahre festgesetzt haben.%
}


%%%%%%%%%%%%%%%%%%%%%%%%%%%%%%%%%%%%%%%%%%%

% time: 11:00
\noindent%
\newTimeslot{11:00}
\abstractMathe{%
  Felix Kunde%
}{%
  Tour de FOSS4G - Eine Reise durch den großen Dschungel freier Software für Geodaten%
}{%
}{%
  Es gibt so viele tolle Open Source Projekte im FOSS4G Umfeld, von denen man als FOSSGIS-Besucher eventuell nichts mitbekommt. Entweder kommen die Kernentwickler nicht aus Deutschland, oder sie setzen mal ein Jahr aus - oder sie mögen einfach nicht vortragen etc.%
}


%%%%%%%%%%%%%%%%%%%%%%%%%%%%%%%%%%%%%%%%%%%

% time: 11:30
\noindent%
\newTimeslot{11:30}
\abstractMathe{%
  Wolfgang Hinsch%
}{%
  Einführung in OpenStreetMap%
}{%
}{%
  Die Entstehung, die heutige Bedeutung und die Vielseitigkeit von OpenStreetMap werden vorgestellt. Es werden Möglichkeiten zur Mitwirkung aufgezeigt.%
}


%%%%%%%%%%%%%%%%%%%%%%%%%%%%%%%%%%%%%%%%%%%

% time: 13:00
\noindent%
\newTimeslot{13:00}
\abstractAudimax{%
  Dominik Helle, Frank Schwarzbach, Dr. Frank Pfeil%
}{%
  Eröffnung der Konferenz 2019%
}{%
}{%
  Eine feierliche Eröffnung der Konferenz durch Vertreter des FOSSGIS e.V. und der HTW Dresden mit wertvollen Hinweisen zum Ablauf und der Organisation.%
}


%%%%%%%%%%%%%%%%%%%%%%%%%%%%%%%%%%%%%%%%%%%

% time: 13:30
\noindent%
\newTimeslot{13:30}
\abstractAudimax{%
  Johannes Terwyen%
}{%
  OSM und öffentliche Verwaltung – Wie geht das?%
}{%
}{%
  Der Regionalverband Ruhr und seine Partner entwickeln zurzeit eine neue Stadtkarte auf der Basis von ALKIS- und OSM-Daten. Nach einer kleinen Einführung in die Inhalte und die Technik beleuchtet dieser Beitrag das Projekt insbesondere unter dem Aspekt der "`Zusammenarbeit"' der Kommunen und der OSM-Community. Beschrieben wird der Prozess des "`Kennenlernens"' und "`Verstehens"'. Im Kern geht es um Respekt, Akzeptanz und Toleranz als Basis für eine fruchtbare Zusammenarbeit zum beiderseitigen Vorteil.%
}


%%%%%%%%%%%%%%%%%%%%%%%%%%%%%%%%%%%%%%%%%%%

% time: 14:05
\noindent%
\newTimeslot{14:05}
\abstractAudimax{%
  Arnulf Christl%
}{%
  Über die Motivation von KI%
}{%
}{%
  "`Artifical Intelligence"', zu Deutsch "`Künstliche Intelligenz"' ist ein großer Begriff, der mit vielen Ängsten, Wünschen und sonstigen Projektionen überladen ist. Lernende Algorithmen dagegen hört sich harmlos an. Sind sie aber nicht, meint ein Ex-Borg und möchte das in einem Lightning Talk erläutern.%
}


%%%%%%%%%%%%%%%%%%%%%%%%%%%%%%%%%%%%%%%%%%%

% time: 14:10
\noindent%
\newTimeslot{14:10}
\abstractAudimax{%
  Christian Mayer%
}{%
  Was sollen diese komischen Adress-Codes?%
}{%
}{%
  Geografische Positionen werden klassischer Weise in (projizierten) Koordinaten angegeben. Mittlerweile gibt es zu diesen Koordinatenangaben mehrere alternative Adressierungssysteme.  Sei es das offene "`Open Location Code"', "`Mapcode"' von einer niederländische Non-Profit-Organisation, das proprietäre "`what3words"' oder die (nicht ganz erst gemeinten) "`what3ducks"' und "`what3ikea"'.
Wer nutzt solche Adresssysteme und braucht die Welt diese überhaupt?%
}


%%%%%%%%%%%%%%%%%%%%%%%%%%%%%%%%%%%%%%%%%%%

% time: 14:15
\noindent%
\newTimeslot{14:15}
\abstractAudimax{%
  Astrid Emde%
}{%
  OSGeo - Teil einer weltweiten Community%
}{%
}{%
  FOSSGIS vertritt D-A-C-H. Doch in Wirklichkeit ist FOSSGIS Teil einer globalen Community. EIne Community die sehr aktiv ist. OSGeo, OSM, FOSS4G finden weltweit und ständig statt.


Impact. Nicht nur Geodaten auch gesellschaftlich...%
}


%%%%%%%%%%%%%%%%%%%%%%%%%%%%%%%%%%%%%%%%%%%

% time: 15:00
\noindent%
\newTimeslot{15:00}
\abstractAudimax{%
  Pirmin Kalberer%
}{%
  Routing mit Open-Source-Software%
}{%
}{%
  pgRouting kennen die Meisten, aber für viele Anwendungsfälle sind andere Open-Source-Tools mindestens ebenso gut geeignet. Dieser Vortrag zeigt Anwendungen vom Eisenbahn- bis zum Schiffsrouting und gibt Tipps zum Einsatz von geeigneten Routing-Technologien.%
}


%%%%%%%%%%%%%%%%%%%%%%%%%%%%%%%%%%%%%%%%%%%

% time: 15:00
\noindent%

\abstractMathe{%
  Christian Clemen%
}{%
  BIM und GIS Interoperabilität%
}{%
}{%
  Der Vortrag vergleicht die Methoden der Informationsverarbeitung der Geo- und Bauwelt und stellt die Zwischenergebnisse einer gemeinsamen Arbeitsgruppe (ISO TC211 und ISO TC59 SC13) "`BIM/GIS Interoperability"' vor.%
}


%%%%%%%%%%%%%%%%%%%%%%%%%%%%%%%%%%%%%%%%%%%

% time: 15:00
\noindent%

\abstractPhysik{%
  Niklas Alt%
}{%
  FOSSGIS im Museum - Eine digitale historische Sozialtopographie%
}{%
}{%
  Der Vortrag stellt eine komplexe Geo-Anwendung (OpenLayers + Angular6) 
vor, die für die Karl-Marx Landesausstellung in Trier entwickelt wurde und im 
dortigen Landesmuseum auf einem großformatigen Bildschirm den Besucher 
einen Einblick in die Unterschicht Triers zur Jugendzeit von Karl Marx 
bot. Neben einer kurzen Einführung des historischen Kontext und 
die technische Umsetzung wird auch das Nutzungsverhalten thematisiert.%
}


%%%%%%%%%%%%%%%%%%%%%%%%%%%%%%%%%%%%%%%%%%%

% time: 15:30
\noindent%
\newTimeslot{15:30}
\abstractAudimax{%
  Peter%
}{%
  GraphHopper Routing Engine - Einblicke und Ausblick%
}{%
}{%
  Der Vortrag wird Einblicke in vergangene und aktuelle Entwicklungen liefern. Auch wird ein Ausblick auf kommende Features nicht fehlen.%
}


%%%%%%%%%%%%%%%%%%%%%%%%%%%%%%%%%%%%%%%%%%%

% time: 15:30
\noindent%

\abstractMathe{%
  Yaseen Srewil%
}{%
  OpenBIM zur Unterstützung der Wohnungswirtschaft, basierend auf einer PostGIS-Datenbank und BIMServer.org%
}{%
}{%
  Die Kombination von BIM-Modellen und Geodaten ist eine Schlüsselfunktion für ein einheitliches digitales Abbild der gebauten Umwelt. Im Rahmen des BMWi-Projektes "`IMMOMATIK"' soll die BIM-Methode auf den Betrieb von Immobilien der Wohnungswirtschaft angewendet werden. Der Fokus liegt auf den Daten, die ausschließlich mit Open Source Software im BIMServer.org und in einer PostGIS Datenbank geführt werden - so können die Vorteile der openBIM-Methodik und Geodaten genutzt werden.%
}


%%%%%%%%%%%%%%%%%%%%%%%%%%%%%%%%%%%%%%%%%%%

% time: 15:30
\noindent%

\abstractPhysik{%
  Lina Dillmann%
}{%
  Usability testing in GIS%
}{%
}{%
  Bedien- und Benutzerfreundlichkeit in Software kann effektiv getestet werden. Welche Testmethoden es gibt und was dabei zu beachten ist wird im Vortrag erklärt.%
}


%%%%%%%%%%%%%%%%%%%%%%%%%%%%%%%%%%%%%%%%%%%

% time: 16:00
\noindent%
\newTimeslot{16:00}
\abstractAudimax{%
  Bernd Marcus%
}{%
  Abseits der öffentlichen Straßen - Eine Routenplanung auf OSM-Basis mit SpatiaLite und QGIS%
}{%
}{%
  Datensätze kommerzieller Navigationssysteme decken Wegenetze außerhalb des öffentlichen Straßennetzes meist nur unzureichend ab. OSM-Daten können diese Lücke in vielen Fällen erfolgreich schließen. Am Beispiel zum Auffinden forstlicher Holzlagerplätze werden die Aspekte zum Aufbau von Routen fähigen Straßendaten erläutert und eine QGIS-Anwendung mit SpatiaLite Unterbau vorgestellt, mit der sich aufgrund fehlender Hausadressierung im Wald die Navigationsstrecke mittels Maus zusammenstellen lässt.%
}


%%%%%%%%%%%%%%%%%%%%%%%%%%%%%%%%%%%%%%%%%%%

% time: 16:00
\noindent%

\abstractMathe{%
  Steffen Hollah, Martin Dresen%
}{%
  Building Information Modeling (BIM) mit Open Source Tools%
}{%
}{%
  Mit den Open Source Tools OpenLayers und Cesium lassen sich Gebäudemodelle in 2D und 3D nebeneinander visualisieren. Das Framework ol-cesium ermöglicht dabei eine einfache Synchronisierung zwischen 2D- und 3D- Ansichten. Im Vortrag werden verschiedene Lösungen und Beispiele gezeigt, die eine Visualisierung von Gebäudemodellen in Kombination mit Geodiensten und Hintergrundkarten ermöglichen.%
}


%%%%%%%%%%%%%%%%%%%%%%%%%%%%%%%%%%%%%%%%%%%

% time: 16:00
\noindent%

\abstractPhysik{%
  Christin Henzen, Lisa Eichler%
}{%
  Ich sehe was, was Du nicht siehst – Die Bewertung der Usability freier WebGIS am Beispiel einer Eyetracking-Studie zum IÖR-Monitor%
}{%
}{%
  Die Usability frei zugänglicher WebGIS variiert derzeit stark. In einer Usability-Studie wurden Usability-Probleme und Verbesserungspotenziale freier WebGIS, am Beispiel des IÖR-Monitors, im Spannungsfeld zwischen subjektiven Nutzerbewertungen und objektiv gemessenen Eyetrackingdaten ermittelt. Die so entstandene Sammlung von Problemen und Lösungsvorschlägen kann während der Entwicklung oder Überarbeitung von WebGIS oder zugrundeliegender APIs zur Verbesserung der Usability genutzt werden.%
}


%%%%%%%%%%%%%%%%%%%%%%%%%%%%%%%%%%%%%%%%%%%

% time: 17:00
\noindent%
\newTimeslot{17:00}
\abstractAudimax{%
  Max Bohnet, Christoph Franke%
}{%
  QGiS-Plugins zum Geocoding und zu intermodaler Erreichbarkeitsanalyse mit dem OpenTripPlanner%
}{%
}{%
  Zwei QGIS-PlugIns werden vorgestellt, die die Funktionalitäten von Geokodierdiensten und von intermodalen Erreichbarkeitsanalysen mit dem OpenTripPlanner in das Desktop-GIS integrieren.%
}


%%%%%%%%%%%%%%%%%%%%%%%%%%%%%%%%%%%%%%%%%%%

% time: 17:00
\noindent%

\abstractMathe{%
  Dirk Stenger%
}{%
  TEAM Engine - Validierung des neuen OGC Standards WFS 3.0 und aktuelle Entwicklungen im Projekt%
}{%
}{%
  Die TEAM Engine ist eine Engine, mit der Entwickler und Anwender Geodienste, wie WFS und WMS, und Geoformate, wie GML oder GeoPackage, testen können.
Dieser Vortrag stellt vor, wie der neue OGC Standard WFS 3.0 mit der TEAM Engine validiert werden kann. Dabei wird der Prozess der Erstellung der neuen Testsuite im Rahmen des OGC Testbed 14-Programms näher beleuchtet. Des Weiteren werden die aktuellen Entwicklungen im TEAM Engine-Projekt aufgezeigt und ein Ausblick gegeben.%
}


%%%%%%%%%%%%%%%%%%%%%%%%%%%%%%%%%%%%%%%%%%%

% time: 17:00
\noindent%

\abstractPhysik{%
  Tim Alder%
}{%
  Pointclouds für OSM%
}{%
}{%
  Der Vortrag beschreibt ein selbstgebautes Geräte zur Laserscandatenerfassung und anschließende Visualisierung mittels Potree.%
}


%%%%%%%%%%%%%%%%%%%%%%%%%%%%%%%%%%%%%%%%%%%

% time: 17:05
\noindent%
\newTimeslot{17:05}
\abstractPhysik{%
  Johannes Kröger%
}{%
  Leider kein LiDAR?%
}{%
}{%
  Ich erzähle kurz zum Stand der Dinge in meinen Versuchen amtliche LiDAR-Daten bzw. das digitale Oberflächenmodell der Stadt Hamburg über das Hamburgische Transparenzgesetz zu befreien.%
}


%%%%%%%%%%%%%%%%%%%%%%%%%%%%%%%%%%%%%%%%%%%

% time: 17:10
\noindent%
\newTimeslot{17:10}
\abstractPhysik{%
  Felix Kunde%
}{%
  GPU Datenbanken%
}{%
}{%
  Was sind GPU-Datenbanken? Wie viel Geo steckt schon drin? Eine kurze Live-Demo mit OmniSci wirds euch zeigen.%
}


%%%%%%%%%%%%%%%%%%%%%%%%%%%%%%%%%%%%%%%%%%%

% time: 17:15
\noindent%
\newTimeslot{17:15}
\abstractPhysik{%
  Alexey Valikov%
}{%
  Preis der Karte%
}{%
}{%
  Wie sehen die Preise der kommerziellen Tile Servern aus?
In diesem Lightning Talk teilen wir die Ergebnisse unserer Analyze des Kartendienst-Marktes.%
}


%%%%%%%%%%%%%%%%%%%%%%%%%%%%%%%%%%%%%%%%%%%

% time: 17:30
\noindent%
\newTimeslot{17:30}
\abstractAudimax{%
  Holger Bruch%
}{%
  Mitfahren-BW - ÖPNV und Fahrgemeinschaften intermodal mit dem OpenTripPlanner%
}{%
}{%
  Freie, intermodale OpenSource-Routenplaner wie OpenTripPlanner, zunehmend als OpenData veröffentlichte ÖPNV-Fahrpläne sowie OpenStreetMap machen neue, innovative Anwendungen möglich, um z.B. Pendlern Alternativen zur Fahrt im eigenen Auto anzubieten.%
}


%%%%%%%%%%%%%%%%%%%%%%%%%%%%%%%%%%%%%%%%%%%

% time: 17:30
\noindent%

\abstractMathe{%
  Matthias Kuhn%
}{%
  QGIS Projektgenerator, vom Datenmodell zur Erfassung%
}{%
}{%
  Das Plugin QGIS Projektgenerator wird vorgestellt. Dieses erlaubt es, aus PostGIS, GeoPackage oder Interlis Datenmodellen ansprechende Erfassungsmasken zu erstellen.
Dabei wird insbesondere auch auf die Herausforderungen eingegangen, die sich ergeben, wenn Daten nach einheitlichem Schema von verschiedenen Stellen erfasst werden sollen.%
}


%%%%%%%%%%%%%%%%%%%%%%%%%%%%%%%%%%%%%%%%%%%

% time: 17:30
\noindent%

\abstractPhysik{%
  Felix Kunde%
}{%
  Räumliche Indizes in PostGIS – Welcher ist der richtige?%
}{%
}{%
  GIST, sp-GIST, BRIN oder BTREE. In PostgreSQL gibt es verschiedene Indextypen, die mittlerweile auch für PostGIS Geometriespalten untersützt werden. Der Vortrag wird kurz die wesentlichen Unterschiede vorstellen und anhand von einfachen Regressionstests mit künstlichen Geodaten aufzeigen, wann welcher Typ an besten verwendet werden sollte.%
}


%%%%%%%%%%%%%%%%%%%%%%%%%%%%%%%%%%%%%%%%%%%

% time: 18:00
\noindent%
\newTimeslot{18:00}
\abstractAudimax{%
  Simon Nieland%
}{%
  "`Urban Mobility Accessibility Computer (UrMoAC)"' – Ein Open Source Tool zur Berechnung von Erreichbarkeitsmaßen%
}{%
}{%
  Der Urban Mobility Accessibility Computer ist ein Open Source Tool zur Berechnung von  urbanen Erreichbarkeitsmaßen. Es wird dazu verwendet Verkehrssysteme hinsichtlich ihrer Effektivität und Verfügbarkeit zu bewerten und stellt somit eine wertvolle Grundlage für städtische Verkehrsplanung dar.%
}


%%%%%%%%%%%%%%%%%%%%%%%%%%%%%%%%%%%%%%%%%%%

% time: 18:00
\noindent%

\abstractMathe{%
  Andreas Schmid%
}{%
  Geodatenmanagement mit GRETL%
}{%
}{%
  Beim Amt für Geoinformation des Kantons Solothurn steht seit 2018 das Datenmanagement-Tool GRETL im Einsatz für den Datenimport und -export in die bzw. aus der PostGIS-Datenbank, aber auch für Datenumbauten von einem Datenmodell in ein anderes. GRETL ist ein Plugin für das *Gradle Build Tool*, wodurch die volle Power eines Build Tools neu für Geodatenflüsse zur Verfügung steht.%
}


%%%%%%%%%%%%%%%%%%%%%%%%%%%%%%%%%%%%%%%%%%%

% time: 18:00
\noindent%

\abstractPhysik{%
  Andreas Kretschmer%
}{%
  PostgreSQL: EXPLAIN erklärt%
}{%
}{%
  Bei der Analyse von Performance-Problemen (Warum ist diese Abfrage langsam) kann eine Auswertung
des Abfrageplanes via EXPLAIN helfen. Doch wie liest man dies?%
}


%%%%%%%%%%%%%%%%%%%%%%%%%%%%%%%%%%%%%%%%%%%

% time: 18:30
\noindent%
\newTimeslot{18:30}
\abstractMathe{%
  Arne Schubert, Stephan Herritsch%
}{%
  YAGA Anwendertreffen%
}{%
}{%
  Das YAGA Entwicklerteam bietet eine Vielzahl von OpenSource Projekten an, mit denen leicht moderne Web-, App- und Server-Anwendungen erstellt werden können. Darunter leaflet-ng2, einer granularen Integration von Leaflet in Angular.io, sowie diverse Docker-Images zum Aufbau einer Geodateninfrastuktur (GDI). Das Anwendertreffen bietet die Möglichkeit zum gegenseitigen Austausch, sowie Hilfestellungen bei Problemen und Fragen.%
}


%%%%%%%%%%%%%%%%%%%%%%%%%%%%%%%%%%%%%%%%%%%

% time: 18:30
\noindent%

\abstractPhysik{%
  Astrid Emde%
}{%
  PostNAS-Suite Anwendertreffen%
}{%
}{%
  Die PostNAS-Suite Anwender treffen sich jedes halbe Jahr zum Austausch. Das nächste Treffen soll auf der FOSSGIS 2019 stattfinden. Hier sollen aktuelle Entwicklungen im PostNAS-Suite Projekt vorgestellt werden. Das Treffen richtet sich besonders an Neueinsteiger, die die Möglichkeiten kennenlernen wollen.%
}


%%%%%%%%%%%%%%%%%%%%%%%%%%%%%%%%%%%%%%%%%%%

% time: 19:00
\noindent%
\newTimeslot{19:00}
\abstractOther{%
  Dominik Helle%
}{%
  Dialoge im Bärenzwinger%
}{%
}{%
  Für die Abendveranstaltung "`Dialoge im Bärenzwinger"' wurde der Bärenzwinger Dresden ausgewählt. Für das Buffet steht außerdem das direkt daran angrenzende historische Kassemattengewölbe zur Verfügung. 

https://www.baerenzwinger.de/

Der Studentenclub liegt direkt an der historischen Altstadt, den Brühlschen Terrassen, und ist per Straßenbahnlinie 3, 7, 8 und 9 erreichbar. Zu Fuß ist die Veranstaltung ab Zentralgebäude etwa 30min entlang der Einkaufsstraße "`Prager Straße"' entfernt.%
}


%%%%%%%%%%%%%%%%%%%%%%%%%%%%%%%%%%%%%%%%%%%
